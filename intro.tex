\section{Introduction}
\label{sec:introduction}

Many social networking, e-commerce, and web properties contain
data-derived products, which usually consist of some data mining
application exposing insights to the user. Typical products include:
``People You May Know,'' a link prediction system attempting to find
others one might know on the social
network~(c.f.~Figure~\ref{fig:pymk}); collaborative filtering, which
showcases relationships between pairs of items based on the wisdom of
the crowd~(c.f.~Figure~\ref{fig:browsemaps}); job and other entity
recommendations; and more. \linkedin\footnote{Anonymized name} is a
top-5 social network with more than 120 million members consisting of
these and twenty other data-derived products. 

The challenge with these data-derived products is that they operate at
a large scale: they must surface hundreds of results for each of our
120 million members. More importantly, due to the dynamic nature of
the social graph, this derived data changes extremely
frequently---requiring an almost complete refresh of the data, all the
while still serving existing traffic with minimal additional latency.

The product data cycle in this context consists of a continuous cycle
of three phases: data collection, processing and finally serving. The
data collection phase usually involves log ingestion, while the
processing phase is a distributed parallel computing infrastructure,
like Hadoop. The end goal of this batch processing is to surface
insights that can then be provided back to the user. 

This paper presents Project \projectname{}, our key-value solution for
the final serving phase of this cycle, and discusses how it fits into
our product ecosystem. The challenge of this system is the frequent
background bulk loading of full data sets, with data refreshes
happening several times a day. At \linkedin, this system has been
running for over 2 years with our largest cluster loading around 3~TB
of new data to the site every day and serving $x$~TB to end users. 

\projectname{} is inspired by Amazon's Dynamo~\cite{dynamo} and was
initially designed to only support the fast online read/write load.
Its extensible storage layer allowed us to quickly build our own
custom read-only storage engine to integrate with the offline data
cycle and support these batch-oriented use cases.

\projectname{} supports instantaneous rollback, where data can be
restored to a clean copy minimizing the time in error, which helps
support fast, iterative development, especially necessary on new
feature improvements. The system is horizontally scalable, and to
enable easy addition of new nodes, the system also supports
rebalancing of existing data. Node failures are handled via
replication.

Our evaluation, and the results we see in production, indicate that
the system supports bulk loads during reads, is horizontally scalable,
and regularly yields sub-20~ms read times---even during data
refreshes.

\projectname{} is open source and readily available under the Apache
license. 

\begin{figure}
\centering
\subfloat[][]{\label{fig:pymk}\dots figure removed for blind review\dots}

\subfloat[][]{\label{fig:browsemaps}\dots figure removed for blind review\dots}

\caption{\subref{fig:pymk}~The ``People You May Know'' module
\subref{fig:browsemaps}~An example collaborative filtering
application.}
\end{figure}

% vim: set ft=tex:
